%%%%%%%%%%%%%%%%%%%%%%%%%%%%%%%%%%%%%%%%%%%%%%%%%%%%%%%%%%%%%%%%%%%
%                                                                 %
%                            APPENDICES                           %
%                                                                 %
%%%%%%%%%%%%%%%%%%%%%%%%%%%%%%%%%%%%%%%%%%%%%%%%%%%%%%%%%%%%%%%%%%%
 
\appendix    % This command is used only once!
\addcontentsline{toc}{chapter}{APPENDICES}             %toc entry  or:
%\addtocontents{toc}{\parindent0pt\vskip12pt APPENDICES} %toc entry, no page #
\chapter{Example LAMMPS Input Cards}

\section{CVFF Input Card}
\label{appendix:cvff}
\begin{singlespace}
\begin{verbatim}
# structure info
boundary p p p
units real
dimension 3
atom_style full
kspace_style pppm 1.0e-6
pair_style lj/cut/coul/long 12.0 8.0
bond_style harmonic
angle_style harmonic
dihedral_style harmonic
improper_style cvff

# Read in montmorillonite data file
read_data Combined_Card.data

# Use Waldman-Hagler combination rule
pair_modify mix sixthpower

# set water group
group water type 16 17

kspace_style pppm 1.0e-6
neigh_modify delay 0 every 1 check yes

# Generate random velocities for all particles T=300K random seed 8746
velocity all create 300 8746

# set timestep
timestep 0.25
thermo 10
fix 2 all nve/limit 0.05
run 1000
unfix	2
fix 2 all npt temp 300 300 10 iso 1 1 100
run 2000
dump 1 water custom 1 waterData id type mol xu yu zu
dump 22 all xyz 100 posout.xyz
run 200000
\end{verbatim}
\end{singlespace}

\section{ReaxFF Input Card}
\label{appendix:reaxff}
\begin{singlespace}
\begin{verbatim}
# structure info
boundary p p p
%units real
dimension 3

# read data
atom_style full
read_data Kaolinite.data
pair_style reax/c NULL
pair_coeff * * ffield_CaSiAlO_Pitman.bin Al H O Si
velocity all create 300 8746

# set timestep
timestep 0.5
thermo 10
fix 1 all qeq/reax 1 0.0 10.0 1.0e-6 reax/c
fix 2 all nve/limit 0.05
dump 1 all xyz 10 posout.xyz
run 1000
unfix	2

fix 2 all npt temp 300 300 10 iso 1 1 100

run 1000

fix 3 all nve 
run 20000
\end{verbatim}
\end{singlespace}


\chapter{Codes Used For Analysis}

\section{Python Script for Fourier Filter and Peak Locator}
\label{appendix:fft_filter}
\begin{singlespace}
\begin{verbatim}
"""
    Fourier Filter
    Hunter Belanger - hunter.belanger@gmail.com
    Perform a FFT to remove high frequency noise from a signal stored
    in a csv file. Fitered signal also output to csv.
"""

# Imports -----------------------------------------------------
import peakutils
import numpy as np
from numpy import fft

# Functions ---------------------------------------------------
def Read_CSV(flnm):
    y = []
    x = []
    # Open file
    fl = open(flnm)
    # Read in data
    for line in fl:
        line = line.split(",")
        x.append(float(line[0]))
        y.append(float(line[1]))
    # Close file
    fl.close()
    
    return (x,y)

def RM_Background(y, linear=True):
    if linear:
        bg = np.arange(y[0], y[-1], (y[-1]-y[0])/float(len(y)))
        if len(bg) != len(y):
            print "Help"
            bg = bg[:-1]
        return y - bg
    
    else:
        bg = []
        for i in range(len(y)):
        bg.append(min(y))
        return np.array(y) - np.array(bg)

def FFT_Filter(x,y,frq):
    F = fft.rfft(y)
    frqs = fft.rfftfreq(len(y), (x[1]-x[0]))
    
    for i in range(len(F)):
        if frqs[i] >= frq:
        F[i] = 0

    y_out = fft.irfft(F)
    
    return y_out

def Write_CSV(x,y,z, flnm):
    fl = open(flnm, "w")

    for i in range(len(x)):
        line = str(x[i]) + "," + str(y[i]) + "," + str(z[i]) + "\n"
        fl.write(line)
    
    fl.close()

# Main --------------------------------------------------------
if __name__ == "__main__":
    flnm = raw_input("Enter File Name => ")
    
    # Read in data
    t,s = Read_CSV(flnm)
    
    wn = []
    ab = []
    
    # Only keep data above 2500 wavenumbers
    for i in range(len(t)):
        if t[i] >= 2500:
        wn.append(t[i])
        ab.append(s[i])
    
    # Removes last data point which is extraneous
    wn = wn[:-1]
    ab = ab[:-1]
    
    # Remove background
    ab = RM_Background(ab)
    
    # Perform Fourier Filter
    ab_out = FFT_Filter(wn,ab,0.024)
    
    # Send data to numpy arrays
    wn = np.array(wn)
    ab_out = np.array(ab_out)
    
    # Hydrx indecies
    lH = np.where( abs(wn - 3350) == min(abs(wn - 3350)))
    hH = np.where( abs(wn - 3450) == min(abs(wn - 3450)))
    
    # Wtr indecies
    lW = np.where( abs(wn - 3450) == min(abs(wn - 3450)))
    hW = np.where( abs(wn - 4000) == min(abs(wn - 4000)))
    
    # Gets Hydrx pk by fitting Gaussians
    pksH = peakutils.interpolate(wn[lH[0][0]:hH[0][0]], 
                                 ab_out[lH[0][0]:hH[0][0]])

    # Gets Hydrx pk by fitting Gaussians
    pksW = peakutils.interpolate(wn[lW[0][0]:hW[0][0]],
                                 ab_out[lW[0][0]:hW[0][0]])
    
    print pksH
    print pksW
    print pksW[0]
\end{verbatim}
\end{singlespace}

\newpage
\section{C++ Program to Obtain Vibrational Modes of Water Molecules}
\label{appendix:cpp_freqs}
\begin{singlespace}
\begin{verbatim}
/*
    Frequency Parser
    Hunter Belanger - hunter.belanger@gmail.com

    Compile: g++ -Ofast -lgsl -lgslcblas -lm -lfftw3 -Wall main.cpp -o main
    Run: ./main path/to/dump/file LAMMPS_Timestpe Data_Dumped_Every 
    (Eg: ./main waterData 0.25 1)
*/

#include <iostream>
#include <iomanip>
#include <math.h>
#include <string>
#include <fstream>
#include <vector>
#include <boost/algorithm/string.hpp>
#include <fftw3.h>
#include <gsl/gsl_rng.h>
#include <gsl/gsl_matrix.h>
#include <gsl/gsl_vector.h>
#include <gsl/gsl_blas.h>
#include <gsl/gsl_multifit_nlinear.h>

using namespace std;
using namespace boost::algorithm;

struct data {
    size_t n;
    double * t;
    double * y;
};

int expb_f (const gsl_vector * x, void *data, gsl_vector * f) {
    size_t n = ((struct data *)data)->n;
    double *t = ((struct data *)data)->t;
    double *y = ((struct data *)data)->y;

    double A = gsl_vector_get (x, 0);
    double x0 = gsl_vector_get (x, 1);
    double w = gsl_vector_get (x, 2);

    size_t i;

    for (i = 0; i < n; i++) {
        /* Model Yi = A * exp(-(t_i - x0)^2/w)*/
        double Yi = A * exp (- ((t[i] - x0)*(t[i] - x0))/w );
        gsl_vector_set (f, i, Yi - y[i]);
    }

    return GSL_SUCCESS;
}

int expb_df (const gsl_vector * x, void *data, gsl_matrix * J) {
    size_t n = ((struct data *)data)->n;
    double *t = ((struct data *)data)->t;

    double A = gsl_vector_get (x, 0);
    double x0 = gsl_vector_get (x, 1);
    double w = gsl_vector_get (x, 2);

    size_t i;

    for (i = 0; i < n; i++) {
        /* Jacobian matrix J(i,j) = dfi / dxj, */
        double e = exp(- (t[i] - x0)*(t[i] - x0)/w);
        gsl_matrix_set (J, i, 0, e);
        gsl_matrix_set (J, i, 1, A * e * (2*(t[i]-x0)/w));
        gsl_matrix_set (J, i, 2, A * e * ((t[i]-x0)*(t[i]-x0))/(w*w));
    }

    return GSL_SUCCESS;
}

void Numbers_From_File(char* &fname, int &ntsteps, int &natoms, 
                       int &minmol, int &maxmol) {
    ifstream dataFile(fname);
    string line;
    vector<string> lineV = vector<string>();
    int count = 0;
    int acount = 0;
    bool parse (false);
    while(getline(dataFile, line)) {
        count += 1;
        if (count == 4) {natoms = stoi(line);}

        if ((ntsteps == 1) and (parse)) {
            acount += 1;
            split(lineV, line, is_space());
            if (stoi(lineV[2]) < minmol) {minmol = stoi(lineV[2]);}
            else if (stoi(lineV[2]) > maxmol) {maxmol = stoi(lineV[2]);}

            if (acount == natoms) {parse = false;}
        }

        if (trim_copy(line) == "ITEM: ATOMS id type mol xu yu zu") {
            ntsteps += 1;
            parse = true;
        }
    }

    dataFile.close();
}

void Sort_Molecules(double** &data, double*** molecules, int* &H1, 
                    int* &H2, int* &O, int &tstep, int &natoms, 
                    int &minmol, int &maxmol) {
    double* rH1 = (double*) calloc(3, sizeof(double));
    double* rH2 = (double*) calloc(3, sizeof(double));
    double* rO = (double*) calloc(3, sizeof(double));
    double* r1 = (double*) calloc(3, sizeof(double));
    double* r2 = (double*) calloc(3, sizeof(double));
    int cnt = 0;
    for (int i = 0; i < (maxmol - minmol + 1); i++) {
        for (int atm = 0; atm < natoms; atm++) {
            if (data[atm][0] == H1[i]) {
                rH1[0] = data[atm][3];
                rH1[1] = data[atm][4];
                rH1[2] = data[atm][5];
                cnt += 1;
            }
            else if (data[atm][0] == H2[i]) {
                rH2[0] = data[atm][3];
                rH2[1] = data[atm][4];
                rH2[2] = data[atm][5];
                cnt += 1;
            }
            else if (data[atm][0] == O[i]) {
                rO[0] = data[atm][3];
                rO[1] = data[atm][4];
                rO[2] = data[atm][5];
                cnt += 1;
            }

            if (cnt == 3) {break;}
        }
        cnt = 0;
        r1[0] = rH1[0] - rO[0];
        r1[1] = rH1[1] - rO[1];
        r1[2] = rH1[2] - rO[2];

        r2[0] = rH2[0] - rO[0];
        r2[1] = rH2[1] - rO[1];
        r2[2] = rH2[2] - rO[2];

        double mr1 = pow( r1[0]*r1[0] + r1[1]*r1[1] + r1[2]*r1[2] ,0.5);
        double mr2 = pow( r2[0]*r2[0] + r2[1]*r2[1] + r2[2]*r2[2] ,0.5);

        r1[0] /= mr1;
        r1[1] /= mr1;
        r1[2] /= mr1;

        r2[0] /= mr2;
        r2[1] /= mr2;
        r2[2] /= mr2;

        double  theta = acos( r1[0]*r2[0] + r1[1]*r2[1] + r1[2]*r2[2] );

        // put values in molecules matrix
        molecules[i][0][tstep] = mr1;
        molecules[i][1][tstep] = mr2;
        molecules[i][2][tstep] = theta;
    }
}

void Get_IDs(double** &data, int* &H1, int* &H2, int* &O, int &natoms, 
             int &minmol, int &maxmol) {
    for (int nmol = minmol; nmol <= maxmol; nmol ++) {
        int idH1 = 0;
        int idH2 = 0;
        int idO = 0;
        int cnt = 0;
        for (int i = 0; i < natoms; i++) {
            if (nmol == data[i][2]) {
                if (data[i][1] == 18) {
                    idO = (int) data[i][0];
                    cnt += 1;
                }
                else {
                    if (idH1 == 0) {
                        idH1 = (int) data[i][0];
                        cnt += 1;
                    }
                    else {
                        idH2 = (int) data[i][0];
                        cnt += 1;
                    }
                }
            }

            if (cnt == 3) {break;}
        }
        H1[nmol - minmol] = idH1;
        H2[nmol - minmol] = idH2;
        O[nmol - minmol] = idO;
    }
}

void Get_Vibrations(char* &fname, double*** &molecules, int &natoms, 
                    int &nmol, int &minmol, int &maxmol) {
    // Arrays for atom ids
    int* H1 = (int*) calloc(nmol, sizeof(int));
    int* H2 = (int*) calloc(nmol, sizeof(int));
    int* O = (int*) calloc(nmol, sizeof(int));

    // Initial line string and vector for boost
    string line ("Initial String");
    vector<string> lineV = vector<string>();

    // Array for timestep data
    double** stepdata = (double**) calloc(natoms, sizeof(double*));
    for (int i = 0; i < natoms; i++) {
        stepdata[i] = (double*) calloc(6, sizeof(double));
    }

    // Open file
    ifstream dataFile(fname);
    bool parse (false);
    int tstep = 0;
    int acount = 0;
    // Go through all lines in file
    while (getline(dataFile, line)) {
        // If in data section of time step
        if (parse) {
            // Add atom data to stepdata array
            split(lineV, line, is_space());
            stepdata[acount][0] = stod(lineV[0]);
            stepdata[acount][1] = stod(lineV[1]);
            stepdata[acount][2] = stod(lineV[2]);
            stepdata[acount][3] = stod(lineV[3]);
            stepdata[acount][4] = stod(lineV[4]);
            stepdata[acount][5] = stod(lineV[5]);
            acount += 1;

            // If all atoms in timestep read
            if (acount == natoms) {
                parse = false;
                if (tstep == 0) {
                    Get_IDs(stepdata, H1, H2, O, natoms, minmol, maxmol);
                }
                Sort_Molecules(stepdata, molecules, H1, H2, O, tstep,
                               natoms, minmol, maxmol);
                tstep += 1;
                acount = 0;
            }
        }

        // If at beginning of timestep
        else if (trim_copy(line) == "ITEM: ATOMS id type mol xu yu zu") {
            parse = true;
        }
    }
}

double Fit_Peaks(double* &x_in, double* &y_in, int &low, int &high) {
    size_t n = high - low + 1;

    const gsl_multifit_nlinear_type *T = gsl_multifit_nlinear_trust;
    gsl_multifit_nlinear_workspace *w;
    gsl_multifit_nlinear_fdf fdf;
    gsl_multifit_nlinear_parameters fdf_params = 
	   gsl_multifit_nlinear_default_parameters();
    const size_t p = 3; // Number of parameters

    gsl_vector *f;
    //gsl_matrix *J;
    gsl_matrix *covar = gsl_matrix_alloc (p, p);
    double* t = (double*) calloc(n,sizeof(double));
    double* y = (double*) calloc(n,sizeof(double));
    double* weights = (double*) calloc(n,sizeof(double));
    struct data d = {n, t, y};

    // Inital values based on which peak
    double x_init[3] = {0, 0, 0};
    if (x_in[low] < 1400) {
        x_init[0] = 0.4;
        x_init[1] = 1440.0;
        x_init[2] = 3400.0;
    }
    else if (x_in[low] < 3600) {
        x_init[0] = 0.8;
        x_init[1] = 3610.0;
        x_init[2] = 1000.0;
    }
    else {
        x_init[0] = 0.8;
        x_init[1] = 3670.0;
        x_init[2] = 1000.0;
    }
    gsl_vector_view x = gsl_vector_view_array (x_init, p);
    gsl_vector_view wts = gsl_vector_view_array(weights, n);
    gsl_rng * r;
    //double chisq;
    double chisq0;
    int info;
    size_t i;

    const double xtol = 1e-8;
    const double gtol = 1e-8;
    const double ftol = 0.0;

    gsl_rng_env_setup();
    r = gsl_rng_alloc(gsl_rng_default);

    /* define the function to be minimized */
    fdf.f = expb_f;
    fdf.df = expb_df;   /* set to NULL for finite-difference Jacobian */
    fdf.fvv = NULL;     /* not using geodesic acceleration */
    fdf.n = n;
    fdf.p = p;
    fdf.params = &d;

    // Define data to be fitted
    for (i = 0; i < n; i++) {
        t[i] = x_in[low+i];
        y[i] = y_in[low+i];
        weights[i] = 1.0;
    }

    /* allocate workspace with default parameters */
    w = gsl_multifit_nlinear_alloc (T, &fdf_params, n, p);

    /* initialize solver with starting point and weights */
    gsl_multifit_nlinear_winit (&x.vector, &wts.vector, &fdf, w);

    /* compute initial cost function */
    f = gsl_multifit_nlinear_residual(w);
    gsl_blas_ddot(f, f, &chisq0);

    /* solve the system with a maximum of 1000 iterations */
    gsl_multifit_nlinear_driver(1000, xtol, gtol, ftol, NULL, NULL, &info, w);

    // Get frequency from fitted parameters
    double x0 = gsl_vector_get(w->x,1);

    gsl_multifit_nlinear_free (w);
    gsl_matrix_free (covar);
    gsl_rng_free (r);

    return x0;
}

void Get_Frequencies(double*** &molecules, vector<double> &fWl, 
                     vector<double> &fWh, vector<double> &fTheta, 
                     double &tstep, int &ntsteps, int &nmols) {
    // Initialize frequency array to used for fitting
    int n_out = (ntsteps/2) + 1;
    double c = 29979245800.0;
    double* freq = (double*) calloc(n_out, sizeof(double));
    int thLow = 0;
    int thHigh = 0;
    int w1Low = 0;
    int w1High = 0;
    int w2High = 0;
    for (int i = 0; i < n_out; i++) {
        freq[i] = (double)i / (tstep*n_out*2);
        freq[i] /= c;
    }

    for (int i = 0; i < n_out; i++) {
        // Get indicies for fitting locations
        if ((freq[i] < 1200) and (freq[i+1] > 1200)) {thLow = i;}
        else if ((freq[i] < 1700) and (freq[i+1] > 1700)) {thHigh = i+1;}

        else if ((freq[i] < 3450) and (freq[i+1] > 3450)) {w1Low = i;}
        else if ((freq[i] < 3625) and (freq[i+1] > 3625)) {w1High = i+1;}

        else if ((freq[i] < 3700) and (freq[i+1] > 3700)) {w2High = i+1;}
    }

    // Go get FFT and peaks for all molecules
    for (int i = 0; i < nmols; i++) {
        // Define input vectors of real values
        double* inr1 = molecules[i][0];
        double* inr2 = molecules[i][1];
        double* inTh = molecules[i][2];

        // Get averages and remove
        double r1avg = 0;
        double r2avg = 0;
        double thtavg = 0;
        for (int j = 0; j < ntsteps; j++) {
            r1avg += inr1[j];
            r2avg += inr2[j];
            thtavg += inTh[j];
        }
        r1avg /= (double) ntsteps;
        r2avg /= (double) ntsteps;
        thtavg /= (double) ntsteps;
        for (int j = 0; j < ntsteps; j++) {
            inr1[j] -= r1avg;
            inr2[j] -= r2avg;
            inTh[j] -= thtavg;
        }


        // Output vectors for complex values
        fftw_complex* outr1 = 
            (fftw_complex*)fftw_malloc(sizeof(fftw_complex)*n_out);
        fftw_complex* outr2 = 
            (fftw_complex*)fftw_malloc(sizeof(fftw_complex)*n_out);
        fftw_complex* outTh = 
            (fftw_complex*)fftw_malloc(sizeof(fftw_complex)*n_out);

        // Define FFT plan for the three functions
        fftw_plan planr1, planr2, planTh;
        planr1 = fftw_plan_dft_r2c_1d(ntsteps, inr1, outr1, FFTW_ESTIMATE);
        planr2 = fftw_plan_dft_r2c_1d(ntsteps, inr2, outr2, FFTW_ESTIMATE);
        planTh = fftw_plan_dft_r2c_1d(ntsteps, inTh, outTh, FFTW_ESTIMATE);

        // Perform FFT
        fftw_execute(planr1);
        fftw_execute(planr2);
        fftw_execute(planTh);

        // Destroy previous plans and in/outs
        fftw_destroy_plan(planr1);
        fftw_destroy_plan(planr2);
        fftw_destroy_plan(planTh);

        // Arrays for frequency magnitudes
        double* magr1 = (double*) calloc(n_out, sizeof(double));
        double* magr2 = (double*) calloc(n_out, sizeof(double));
        double* magth = (double*) calloc(n_out, sizeof(double));

        double maxr1 = 0;
        double maxr2 = 0;
        double maxth = 0;
        // Add magnitudes to arrays
        for (int j = 0; j < n_out; j++) {
            magr1[j] = pow( outr1[j][0]*outr1[j][0] + 
                            outr1[j][1]*outr1[j][1] ,0.5);
            if (magr1[j] > maxr1) {maxr1 = magr1[j];}
            magr2[j] = pow( outr2[j][0]*outr2[j][0] + 
                            outr2[j][1]*outr2[j][1] ,0.5);
            if (magr2[j] > maxr2) {maxr2 = magr2[j];}
            magth[j] = pow( outTh[j][0]*outTh[j][0] + 
                            outTh[j][1]*outTh[j][1] ,0.5);
            if (magth[j] > maxth) {maxth = magth[j];}
        }
        // Normalize arrays
        for (int j = 0; j < n_out; j++) {
            magr1[j] /= maxr1;
            magr2[j] /= maxr2;
            magth[j] /= maxth;
        }

        // Fit curves and get frequencies
        double freqTht = Fit_Peaks(freq, magth, thLow, thHigh);
        if ( (freqTht < 1700) and (freqTht > 1200) ) 
        {fTheta.push_back(freqTht);}

        double freqLow = Fit_Peaks(freq, magr1, w1Low, w1High);
        if ( (freqLow < 3625) and (freqLow > 3450) ) 
        {fWl.push_back(freqLow);}

        freqLow = Fit_Peaks(freq, magr2, w1Low, w1High);
        if ( (freqLow < 3625) and (freqLow > 3450) ) 
        {fWl.push_back(freqLow);}

        double freqHigh = Fit_Peaks(freq, magr1, w1High, w2High);
        if ( (freqHigh < 3700) and (freqHigh > 3625) ) 
        {fWh.push_back(freqHigh);}

        freqHigh = Fit_Peaks(freq, magr2, w1High, w2High);
        if ( (freqHigh < 3690) and (freqHigh > 3650) ) 
        {fWh.push_back(freqHigh);}

        fftw_free(outr1);
        fftw_free(outr2);
        fftw_free(outTh);
    }
}

int main (int argc, char* argv[]) {
    double timestep = stod(argv[2]);
    double outputstep = stod(argv[3]);

    // Get number of atoms and timesteps
    int ntimesteps = 0;
    int natoms = 0;
    int minmol = 1000000;
    int maxmol = 0;
    cout << "Getting numbers...\n";
    Numbers_From_File( argv[1], ntimesteps, natoms, minmol, maxmol);
    cout << "Ntimesteps:" << ntimesteps << " " << "Natoms:" << natoms 
         << " " << "MinMol:" << minmol << " " << "MaxMol:" << maxmol << "\n\n";
    int nmol = maxmol - minmol +1;

    // Initialize array for data to be transformed
    double*** moleculeData = (double***) calloc(nmol, sizeof(double**));
    for (int i = 0; i < nmol; i++) {
        moleculeData[i] = (double**) calloc(3, sizeof(double*));
        for (int j = 0; j < 3; j++) {
            moleculeData[i][j] = (double*) calloc(ntimesteps, sizeof(double));
        }
    }

    // Get molecule data
    cout << "Geting atom data...\n\n";
    Get_Vibrations(argv[1], moleculeData, natoms, nmol, minmol, maxmol);

    cout << "Doing Fourier transforms...\n\n";
    vector<double> fTheta = vector<double>();
    vector<double> fWl = vector<double>();
    vector<double> fWh = vector<double>();
    double tstep = timestep*outputstep*(1E-15); // Convert to s
    Get_Frequencies(moleculeData, fWl, fWh, fTheta, tstep, ntimesteps, nmol);


    // Calculate average frequencies
    double thavg = 0;
    double wlavg = 0;
    double whavg = 0;

    for (size_t i = 0; i < fTheta.size(); i++) {
        thavg += fTheta[i];
    }
    thavg /= (double) fTheta.size();

    for (size_t i = 0; i < fWl.size(); i++) {
        wlavg += fWl[i];
    }
    wlavg /= (double) fWl.size();

    for (size_t i = 0; i < fWh.size(); i++) {
        whavg += fWh[i];
    }
    whavg /= (double) fWh.size();

    // Calculate standard deviations
    double thstd = 0;
    double wlstd = 0;
    double whstd = 0;
    for (size_t i = 0; i < fTheta.size(); i++) {
        thstd += pow( thavg - fTheta[i] ,2);
    }
    thstd /= ((double) fTheta.size() - 1.0);
    thstd = pow(thstd, 0.5);

    for (size_t i = 0; i < fWl.size(); i++) {
        wlstd += pow( wlavg - fWl[i] ,2);
    }
    wlstd /= ((double) fWl.size() - 1.0);
    wlstd = pow(wlstd, 0.5);

    for (size_t i = 0; i < fWh.size(); i++) {
        whstd += pow( whavg - fWh[i] ,2);
    }
    whstd /= ((double) fWh.size() - 1.0);
    whstd = pow(whstd, 0.5);

    // Print averages
    cout << fixed;
    cout << setprecision(3);
    cout << "Tht:   " << thavg << " +/- " << thstd << "\n";
    cout << "Wlow:  " << wlavg << " +/- " << wlstd << "\n";
    cout << "Whigh: " << whavg << " +/- " << whstd << "\n\n";
        
    return 0;
}
\end{verbatim}
\end{singlespace}

\chapter{Supplemental Files}
The supplemental file JACS\_Reuse\_Permission.jpg (252.6 kB), is a jpeg image of the permissions for the reuse of Figure~\ref{fig:clay_structure} in this thesis.
 
