%%%%%%%%%%%%%%%%%%%%%%%%%%%%%%%%%%%%%%%%%%%%%%%%%%%%%%%%%%%%%%%%%%% 
%                                                                 %
%                            ABSTRACT                             %
%                                                                 %
%%%%%%%%%%%%%%%%%%%%%%%%%%%%%%%%%%%%%%%%%%%%%%%%%%%%%%%%%%%%%%%%%%% 
 
\specialhead{RÉSUMÉ}
 
Le sujet de cette thèse est un étude de spectroscopie infrarouge de l’eau intercalé dans les silicates en couches en fonction du pH. C’est une étude importante car les propriétés des argiles, comme le montmorillonite, ont une dépendance sur le pH du système, et la quantité d’eau intercalé dans l’argile. Cela pourrait nous donner de nouvelles connaissances sur ce mécanisme de catalysation, et les interactions d’eau à l’échelle nanométrique.

Une méthode pour contrôler l'hydratation d’échantillons d’argile montmorillonite a été examinée. Les solutions de sel saturées maintiennent l'humidité dans un espace clos. Les échantillons ont été mis dans un dessiccateur, avec la solution de sel saturée qui maintenait une humidité prévue. La diffractométrie de rayons X a été utilisée afin de mesurer l’espace entre les feuillets d’argile, qui est déterminé par la quantité d’eau intercalé, c’est-à-dire le nombre de strates d’eau entre deux feuillets. Les échantillons d’argile de la forme d’un disque comprimé étaient plus difficile à hydrater que l’argile en poudre. C’était aussi évident que l’on n’aurait que quelques minutes pour travailler avec l’échantillon en dehors du dessiccateur avant qu’il ne se retrouve dans un état moins hydraté.

La spectroscopie infrarouge a été employée pour analyser la dépendance du pH sur les fréquences des vibrations moléculaires de l’eau intercalé dans l’argile. Les spectres des échantillons avec des espèces de cations échangeables différents, et venant de mines différentes, ont été obtenus avec des pH entre 3 et 10. Pour chaque cation et mine considéré, la fréquence de l’étirement symétrique d’eau augmentait avec le pH. Il n’y avait pas de changement dans la fréquence de cisaillement d’eau, ni dans la fréquence de l’étirement de la liaison O-H des hydroxyles de l’argile.

Puis, un modèle pour simuler ce système avec la dynamique moléculaire a été réalisé. Utilisant la suite de InterfaceFF, les structures de montmorillonite inclus ont été combinées avec le modèle d’eau flexible. Cela a permit la simulation des vibrations dans les molécules d’eau. Pour obtenir la fréquence de chaque vibration, une transformation de Fourier a été faite sur les positions des atomes, permettant l'obtention des fréquences des vibrations de l’eau. Ces fréquences étaient proche de celles déjà obtenues lors des expériences précédentes.

 
