%%%%%%%%%%%%%%%%%%%%%%%%%%%%%%%%%%%%%%%%%%%%%%%%%%%%%%%%%%%%%%%%%%% 
%                                                                 %
%                            ABSTRACT                             %
%                                                                 %
%%%%%%%%%%%%%%%%%%%%%%%%%%%%%%%%%%%%%%%%%%%%%%%%%%%%%%%%%%%%%%%%%%% 
 
\specialhead{ABSTRACT}

The topic of this thesis is an infrared spectroscopic study of water intercalated in layered silicates as a function of pH. This is important due to the catalytic properties of phyllosilicate clay minerals such as montmorillonite, which have a known dependence on the pH of the system, and the amount of water intercalated in the clay. This could improve our understanding of the catalysation mechanism at play, and the interactions of water at the nanoscale.

For this purpose, a simple method to control the hydration of montmorillonite clay samples was investigated, using saturated salt solutions to control the relative humidity of the storage environment. Samples of montmorillonite were placed in desiccators with different solutions in the base, controlling the ambient moisture level around the samples. X-ray diffraction was used to measure the basal spacing of the samples, indicating the number of layers of water present between clay sheets. It was found that samples in the form of compressed pellets were much more difficult to hydrate than clay powder. The working time outside of the humid environment was also found to be on the order of just minutes, before the clay returns to a lesser hydrated state. 

Infrared spectroscopy was used to investigate the pH dependence of the vibrational modes of water intercalated in the clay. Spectra of montmorillonite with exchangeable cation species of either Li, Na, or K, from differing origin clay mines were collected at pH values between 3 and 10. For all investigated exchangeable cations and clay mines, it was observed that the frequency of the symmetric stretching mode increased as the pH of the system increased. No significant changes in frequency were observed in the water deformation mode, or the O-H stretching mode of the structural hydroxyl groups in the clay.

Subsequently, a model to perform molecular dynamics simulations of the clay-water interface was developed for the interpretation of the experimental data. Using the InterfaceFF suite provided montmorillonite structures were combined with the flexible water model. This allowed for a basic simulation of the water vibrational modes. An algorithm for determining the frequency of these vibrational modes using Fourier transforms on the atom positions is also proposed. Simulations conducted using this system had water vibrational modes similar to those of the experimentally observed frequencies.